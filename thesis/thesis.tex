%Erzeugt mit dem LaTeX-Generator: http://latex.sehnot.de

%Schriftgröße, Layout, Papierformat, Art des Dokumentes
\documentclass[12pt,oneside,a4paper]{scrbook}

%Einstellungen der Seitenränder
\usepackage[left=3cm,right=2cm,top=2cm,bottom=2cm,includeheadfoot]{geometry}

%neue Rechtschreibung
%\usepackage[ngerman]{babel}

%Umlaute ermöglichen
% \usepackage{fontspec}
% \setromanfont{TeX Gyre Pagella}

% \usepackage[math-style=iso]{unicode-math}
% \setmathfont{Asana Math}

\usepackage{graphicx}
\usepackage{wrapfig}
\usepackage[utf8x]{inputenc}
\usepackage{url}
\usepackage{multirow}
\usepackage{amsmath}
\usepackage{hyperref}
\usepackage{multicol}
\usepackage{amsthm}
\usepackage[official]{eurosym}
\usepackage{setspace}
\newtheorem{definition}{Definition}
\newtheorem{satz}{Theorem}
\theoremstyle{definition}
%Zitieren
\usepackage[longnamesfirst, authoryear]{natbib}
\usepackage{minted}
\usepackage{booktabs}



%Kopf- und Fußzeile
\usepackage{fancyhdr}
\pagestyle{fancy}
\fancyhf{}

\newcommand{\HRule}{\rule{\linewidth}{0.5mm}}

%Kopfzeile rechts bzw. außen
\fancyhead[R]{\nouppercase{\leftmark}}
%Linie oben
\renewcommand{\headrulewidth}{0.5pt}

%Fußzeile links bzw. innen
\fancyfoot[L]{Units of Measure - a Scala Macro System}
%Fußzeile rechts bzw. außen
\fancyfoot[R]{\thepage}
%Linie unten
\renewcommand{\footrulewidth}{0.5pt}

\clubpenalty = 50000
\widowpenalty = 50000




\title{Units of Measure - a Scala Macro System}
\author{Julian Schrittwieser}
\date{\today}

\begin{document}


\begin{titlepage}

\begin{center}


% Upper part of the page
\includegraphics[width=0.4\textwidth]{tu-logo.png}\\[3cm]


\textsc{\Large Software \& Information Engineering}\\[0.5cm]


% Title
\HRule \\[0.8cm]
{ \huge \bfseries Units of Measure}\\[0.4cm]
\Large A Scala Macro System

\HRule \\[1.5cm]

\vfill


% Author and supervisor
\begin{minipage}{0.4\textwidth}
\begin{flushleft} \large
\emph{Author:}\\
Julian \textsc{Schrittwieser}
\end{flushleft}
\end{minipage}
\begin{minipage}{0.4\textwidth}
\begin{flushright} \large
\emph{Advisor:} \\
Univ.-Prof. Dr. Jens \textsc{Knoop}
\end{flushright}
\end{minipage}

\vfill

% Bottom of the page
{\large Februar 12th, 2013}

\end{center}

\end{titlepage}

\pagenumbering{roman} % Roman numerals, other styles possible: http://www.image.ufl.edu/help/latex/intext.shtml
\setcounter{page}{2}

% \onehalfspacing
\chapter*{Preface}

Ipsum Lorem
\\ \\
Vienna, February 12th, 2010  \hfill Julian Schrittwieser

\singlespacing
\tableofcontents
%\onehalfspacing

\chapter{Introduction}
\pagenumbering{arabic} % Roman numerals
\setcounter{page}{1}

\section{Motivation}
mars rover example

\section{Nomenclature}
Throughout this paper, I use \emph{Measure} to refer to a number annotated with its corresponding units. The exact implementation of this annotation depends on the context, some libraries use static type informations, others fields in an object. \emph{Units of Measure} is used to refer to any system that provides such an implementation, including methods to perform arithmetic with \emph{Measure}s.

\section{Source Code}
The source code for all benchmarks mentioned can be found on GitHub: \url{https://github.com/Mononofu/Units-of-Measure/tree/master/thesis/benchmarks}. This same repository also contains the source for this document (\url{https://github.com/Mononofu/Units-of-Measure/tree/master/thesis}) as well as the implementation of the Units of Measurement system in Scala (\url{https://github.com/Mononofu/Units-of-Measure}).

\chapter{Related Work}

In principle, there are two distinct ways to implement units of measurement in a programming language: Either statically as part of the type system, or dynamically with objects that know their own unit. Each approach has its own merits and disadvantages, so I examine a few existing implementations in common languages before creating my own solution.

\section{Dynamic Systems}

Some programming languages do not give their users much choice, because they either completely lack static type systems (dynamic languages like Python or Ruby) or their type system is not sophisticated enough for a units of measurement library. In general, those dynamic systems allow access to unit information at runtime but in turn pay a heavy performance price: Each number is stored as a full blown object (see figure \ref{code:naive_java_measure}), not as a simple primitive like a normal number. Thus, all calculations incur additional memory lookups and values take additional space.

\begin{figure}
\begin{minted}{java}
class Measure<T extends java.lang.Number> {
  T      value;
  Unit   unit;
}
\end{minted}
\caption{A naive implementation of a number annotated with its unit in Java. Actual size depends on the number used, but at least 8 bytes are used by the object itself, and further 4/8 bytes (32/64 bit system, resp.) are needed for the reference to the unit.}
\label{code:naive_java_measure}
\end{figure}



\subsection{Java - JScience}

Java has a standardized Units of Measurement API \citep{Units13} with several implementations, both open source and commercial. I will focus on JScience here \citep{Dautelle11}. In addition to support for physical units, JScience also contains libraries for simple mathematical functions, currencies and a linear algebra module. Here, I only focus on parts pertaining to Units of Measurement.

JScience represents Measures using the class \verb/Amount/, but users are free to create their own implementations of \verb/Measurable/. Measures are created with a flexible static helper function (\verb/valueOf/) from a range of different inputs. For details, see the short usage example below in Figure \ref{code:jscience_example}. Essentially, \verb/Amount/ is a wrapper class that tags a number with its corresponding unit - as shown in \ref{code:naive_java_measure} - and provides a set of helper and conversion functions.

\begin{figure}
\begin{minted}{java}
Amount<Mass> m0            = Amount.valueOf(100, POUND);
Amount<ElectricCurrent> m1 = Amount.valueOf("234 mA").to(MICRO(AMPERE));

// m0 =    100 lb
// m1 = 234000 uA

Unit<Mass> u  = m0.getUnit();
Dimension d   = u.getDimension();

\end{minted}
\caption{Example for JScience from the official documentation \citep{Dautelle11}. Usage is straightforward and unit information is still available at runtime.}
\label{code:jscience_example}
\end{figure}

Unit safety is enforced both by the type system at compile time (Measures are generic classes with their units as type parameters) and comparison functions at run time (\verb/equals/ only returns true if both the value and the interval match, etc.).

Unfortunately, the performance hit due to the wrapper classes is quite severe - I experienced a slowdown of around 80 on my machine in a simple benchmark, compared to normal primitives (\verb/int/ and \verb/double/) (see Figure \ref{bench:jscience}).

At the moment, Units of Measurement Systems in Java are only practicable in systems that are not performance critical. Alas, many systems that would benefit most are performance critical - computer game engines, high frequency trading, etc. Therefore, Units of Measurements remain a niche application in Java.

\begin{figure}
\begin{tabular}{lrrr}
method          & int    & double  & JScience \\
\midrule
addition        & 1.0 ns &  1.8 ns    &   149.5 ns \\
multiplication  & 1.1 ns &  1.9 ns    &   162.3 ns
\end{tabular}
\caption{A simple benchmark of JScience compared to primitives data types. JScience is at least 80 times slower on average.}
\label{bench:jscience}
\end{figure}


\subsection{Python - units}

Python is a dynamic language, so the only way to make a system of measurements work is to create a dynamic system that relies on tagged numbers. That's exactly what \emph{units} \citep{Donohue12} does. Usage is very simple, as seen in Figure \ref{code:python_units}. Thanks to operator overloading and the dynamic nature of Python code, units can be introduced into legacy code with only a few changes.

\begin{figure}
\begin{minted}{python}
metre, second = unit('m'), unit('s')                   # define units
print(metre(10) / second(2))                           # 5 m / s
newton = named_unit('N', ['kg', 'm'], ['s', 's'], 1)   # 1 N = 1 kg*m/s^2
units.predefined.define_units()                        # SI units and more
\end{minted}
\caption{\emph{units} makes it very simple to define custom units and to add conversions to existing units, while also including all standard units for immediate use. Calculation happens just like with primitives, thanks to operator overloading.}
\label{code:python_units}
\end{figure}



\begin{figure}
\begin{tabular}{lrrr}
method          & int    & double  & units \\
\midrule
addition        & 151 ns &  164 ns    &    4.136 ns \\
multiplication  & 159 ns &  165 ns    &   22.352 ns
\end{tabular}
\caption{Even though Python is an interpreted language, there is a large difference between primitive numbers, and those provided by the \emph{units} library. The slowdown is especially severe for the multiplication as a new unit has to be generated for the result.}
\label{bench:python_units}
\end{figure}


\section{Static Systems}

Static systems are the complete opposite compared to dynamic systems: They usually do not provide any unit information at runtime, but also do not incur any performance overhead - all units are checked at compile time. To achieve this, they rely on a powerful type system or even compiler plugins, effectively offloading much of the work on the compiler.

\subsection{F\#}

\verb/F#/ is one of only a few languages that have integrated a system of units into the language itself. This allows for easy usage and definition of units since they can rely on special syntax.
\citep{Kennedy08:1}

As usual, \verb/F#/ comes with all common units - both SI and Imperial - predefined \citep{Kennedy08:2}, but it's also easy to interface with code that is not unit aware: Primitive types like \verb/float/ are just a type alias for \verb/float<1>/ (a float with unit dimension).

Thanks to the language support, units also work with generics, including full type (unit) inference. As an example, if we define a simple function \verb/let sqr(x: float<_>) = x*x/, \verb/F#/ will automatically infer that it takes any \verb/float<'u>/ and returns a \verb/float<'u^2>/ \citep{Kennedy08:3}. This works for types of arbitrary complexity, making the programmers life much simpler.

The support for generics also extends to custom types - it's enough to annotate a type parameter with \verb/[<Measure>]/ to make it a unit parameter \citep{Kennedy08:4}. For a example, see Figure \ref{code:fsharp}.


\begin{figure}
\begin{minted}{fsharp}
[<Measure>] type kg                 // define a unit
let gravity = 9.81<m/s^2>           // use units
// generic type
type Vector2< [<Measure>] 'u > = { x: float<'u>, y: float<'u> }
\end{minted}
\caption{Usage examples for the unit system in F\#.}
\label{code:fsharp}
\end{figure}

\subsection{C++ - Boost.Units}

The \verb/Boost.Unit/ library takes advantage of the power of C++ templates in the form of the Boost Metaprogramming Library (MPL).  It relies on the compiler to adequately optimize the code, but then incurs no performance overhead at runtime \citep{Schabel10}.However, since it heavily relies on template metaprogramming it has stringent requirements for the compiler and only works on new, standard compliant compilers.

Variable definitions work as always (\verb/quantity<length> L = 2.0*meters;/), and thanks to operator overloading arithmetic is exactly the same as for primitives.

\verb/Boost.Unit/ is also independent of any single unit systems - functions can be defined generically, for arbitrary unit systems, using just the types of the dimensions. For an example, see Figure \ref{code:boost_units_generic}. Users can then define their own unit systems to interface with already existing library code.

\begin{figure}
\begin{minted}{c++}
template<class System,class Y>
quantity<unit<energy_dimension,System>,Y>
work(quantity<unit<force_dimension,System>,Y> F,
     quantity<unit<length_dimension,System>,Y> dx)
{
    return F*dx;
}
\end{minted}
\caption{The physical definition of work - computed for an arbitrary unit system, from \citep{Schabel10}.}
\label{code:boost_units_generic}
\end{figure}


\subsection{Haskell - Dimensional}
Haskell's \verb/Dimensional/ module \citep{Buckwalter06} completely relies on the type system and is the most elegant of the unit systems presented here. It relies on a single data type (\verb/Dimensional/, see Figure \ref{code:haskell_dimensional}), parameterized with the exponents of different physical dimensions (Figure \ref{code:haskell_dim}). While only integer exponents are possible, rational exponents can be avoided in practice.

\begin{figure}
\begin{minted}{haskell}
newtype (Variant v, Dims d)
      => Dimensional v d a = Dimensional a deriving (Show, Eq, Ord)
\end{minted}
\caption{Dimensional, the unit system's fundamental data type. Since 'a', representing the underlying number, is the only non-phantom type Dimensional is defined as a newtype and does not incur any runtime overhead. From \citep{Buckwalter06}.}
\label{code:haskell_dimensional}
\end{figure}



\begin{figure}
\begin{minted}{haskell}
data (NumType l,    -- Length.
      NumType m,    -- Mass.
      NumType t,    -- Time.
      NumType i,    -- Electric current.
      NumType th,   -- Thermodynamic temperature.
      NumType n,    -- Amount of substance.
      NumType j)    -- Luminous intensity.
   => Dim l m t i th n j

type DLength      = Dim Pos1 Zero Zero Zero Zero Zero Zero
type DForce       = Dim Pos1 Pos1 Neg2 Zero Zero Zero Zero
\end{minted}
\caption{The type Dim, containing the physical base dimensions, and two derived dimesions, from \citep{Buckwalter06}.}
\label{code:haskell_dim}
\end{figure}


As with the other systems, operator overloading allows for easy calculations. \verb/*~/ and \verb|/~| can be used to attach units to numbers and remove them again.


\begin{figure}
\begin{minted}{haskell}
escapeVelocity :: (Floating a) => Mass a -> Length a -> Velocity a
escapeVelocity m r = sqrt (two * g * m / r)
  where
    two = 2 *~ one
    g = 6.6720e-11 *~ (newton * meter ^ pos2 / kilo gram ^ pos2)
\end{minted}
\caption{An example function using the Dimensional module, from \citep{Buckwalter06}.}
\label{code:haskell_dimensional}
\end{figure}

\verb/Dimensional/ is restricted to the physical dimensions mentioned in Figure \ref{code:haskell_dim}. To add more dimensions, the fundamental data type would have to be changed - unfeasible in practice.


\subsection{Scala}

Scala also has a long history of attempts to introduce Units of Measure. One of the oldest of such attempts is a now defunct compiler plugin \citep{Nygard09}. It allowed the programmer to introduce units using multiplication (\verb/400.0*m*m/) and caught common errors when combining units in invalid ways. Sadly, it's only compatible with Scala up to version 2.7.1 - as of 2013, the most recent version is 2.10.0.

Another such example comes from \citep{McBeath08}, who implements a simple unit system based on Church Numerals. However, just as Haskell's \verb/Dimensional/ it is restricted to predefined units and does not allow the user to define her own units.

Finally, there is \verb/ScalaQuantity/ \citep{Hans12}, which is closely modeled on Haskell's \verb/Dimensional/. Again, it does not allow user defined units and is otherwise so similar to \verb/Dimensional/ that I won't cover it in detail.

Currently, there is no Units of Measure system in Scala that allows user defined units or provides for calculations without runtime overhead. This inspired me to create my own library, which I will cover in the following.

\chapter{Objectives}

no runtime overhead

custom units

static unit checking

\chapter{Limitations}

Measure can't be an instance of Numeric, since * returns a new type, and even + might do so

runtime equality checks can't enforce units, since those are not available

\chapter{Acknowledgements}
scala mailing list

#scala on freenode

scala user group vienna


\singlespacing
\bibliographystyle{dinat}
\bibliography{thesis}{}
\addcontentsline{toc}{chapter}{A \;References}

\chapter*{}
\onehalfspacing
Ich erkläre, dass ich diese Fachbereichsarbeit ausschließlich selbst und ohne Gebrauch unerlaubter Hilfsmittel oder Hilfen verfasst habe.



\end{document}
